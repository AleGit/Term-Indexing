% !TeX root = ../m3Handout.tex
% !TeX encoding = UTF-8
% !TeX spellcheck = en_US

%% ==============================================================
%% ### MISC ###

%\newcommand{\EQ}{\simeq}
%\newcommand{\NEQ}{\not\simeq}
\newcommand{\foEQ}{\approx}		%	{\simeq}
\newcommand{\foNEQ}{\not\foEQ}

\newcommand\TOP[2]{\genfrac{}{}{0pt}{}{#1}{#2}}
\newcommand\TOPTEXT[2]{\TOP{\text{#1}}{\text{#2}}}
%\newcommand{\mygreek}[1]{\selectlanguage{polutonikogreek}#1\selectlanguage{english}}
%\newcommand{\mygreek}[1]{{\selectlanguage{polutonikogreek}#1}\selectlanguage{english}}
%\renewcommand{\mygreek}[1]{\foreignlanguage{polutonikogreek}{#1}}

%\newcommand{\iSUB}[2]{#2\!\mapsto\!#1}
%\newcommand{\BgSyntaxTree}{\usebackgroundtemplate{\transparent{0.1}\includegraphics[width=\paperwidth]{SyntaxTreeBackground.png}}}

\newcommand{\EMPH}[1]{\emph{\textcolor{colEm}{#1}}}

%% ==============================================================
%% ### MY MATH ENVIRONMENTS ###

\theoremstyle{plain}
%\newtheorem{theorem}{Theorem}
%\newtheorem{proposition}{Proposition}
%\newtheorem{lemma}{Lemma}
%\newtheorem*{corollary}{Corollary}

\theoremstyle{definition}
%\newtheorem{definition}{Definition}
\newtheorem{Conjecture}{Conjecture}
%\newtheorem*{example}{Example}
%\newtheorem{algorithm}{Algorithm}
\newtheorem{procedure}{Procedure}
\newtheorem{goal}{Goal}
\newtheorem{notation}{Notation}

\theoremstyle{remark}
%\newtheorem*{remark}{Remark}
\newtheorem*{observation}{Observation}
%\newtheorem*{note}{Note}
%\newtheorem{case}{Case}

%% ==============================================================
%% ### MY MATH DEFINITIONS ###

% math operators

%\DeclareMathOperator{\arity}{arity}
%\DeclareMathOperator{\var}{var}
%\DeclareMathOperator{\pos}{pos}
%\DeclareMathOperator{\T}{T}
%\DeclareMathOperator{\dom}{dom}
%\DeclareMathOperator{\rng}{rng}
%\DeclareMathOperator{\img}{img}
\DeclareMathOperator{\mgu}{mgu}	% most general unifier
%\DeclareMathOperator{\wgt}{W\!}
%\DeclareMathOperator{\sel}{sel}
%\DeclareMathOperator{\mul}{mul}
%\DeclareMathOperator{\add}{add}

\DeclareMathOperator{\UNIF}{unifiable}
\DeclareMathOperator{\INST}{instance}
\DeclareMathOperator{\GNRL}{generalization}
\DeclareMathOperator{\VRNT}{variant}
\DeclareMathOperator{\PSTR}{pstr}

\newcommand{\NGTPREQ}{\not\succcurlyeq}

% constant (function, predicate) symbols

\newcommand{\mf}{{\mathsf f}}
\newcommand{\mg}{{\mathsf g}}
\newcommand{\mh}{{\mathsf h}}
\newcommand{\ma}{{\mathsf a}}
\newcommand{\mb}{{\mathsf b}}
\newcommand{\mc}{{\mathsf c}}
\newcommand{\md}{{\mathsf d}}
\newcommand{\mx}{{\mathsf x}}
\newcommand{\my}{{\mathsf y}}
\newcommand{\msucc}{{\mathsf s}}
\newcommand{\mpred}{{\mathsf p}}
\newcommand{\mA}{{\mathsf A}}
\newcommand{\mB}{{\mathsf B}}
\newcommand{\mP}{{\mathsf P}}
\newcommand{\mQ}{{\mathsf Q}}

% caligraphic symbols

\newcommand{\mcC}{{\mathcal C}}
\newcommand{\mcD}{{\mathcal D}}
%\newcommand{\mcE}{{\mathcal E}}
%\newcommand{\mcF}{{\mathcal F}}
%\newcommand{\mcM}{{\mathcal M}}
%\newcommand{\mcO}{{\mathcal O}}
\newcommand{\mcP}{{\mathcal P}}
%\newcommand{\mcR}{{\mathcal R}}
%\newcommand{\mcT}{{\mathcal T}}
\newcommand{\mcV}{{\mathcal V}}

\newcommand{\Var}{{}\mcV\mathsf{ar}}
\newcommand{\Dom}{{}\mcD\mathsf{om}}
\newcommand{\Pos}{{}\mcP\mathsf{os}}
\newcommand{\PosStr}{\Pos^\Sigma}

% fraktal symbols

\newcommand{\mfC}{{\mathfrak C}}
\newcommand{\mfL}{{\mathfrak L}}
\newcommand{\mfR}{{\mathfrak R}}
\newcommand{\mfT}{{\mathfrak T}}

\newcommand{\SIGA}{{\mathcal A}}
\newcommand{\SIGC}{{\mathcal C}}
\newcommand{\SIGE}{{\mathcal E}}
\newcommand{\SIGF}{{\mathcal F}}
\newcommand{\SIGL}{{\mathcal L}}
\newcommand{\SIGP}{{\mathcal P}}
\newcommand{\SIGS}{{\mathcal S}}
\newcommand{\SIGT}{{\mathcal T}}
\newcommand{\SIGV}{{\mathcal V}}
% tt symbols

\newcommand{\mtS}{{\mathtt S}}
\newcommand{\sgr}{\succ_{\mathsf gr}}

% 

\newcommand{\TI}[1]{^{^{#1:}}\!}
\newcommand{\ANGLES}[1]{\langle#1\rangle}

\newcommand{\joins}{\rightarrow^*\cdot^*\!\!\leftarrow}
\newcommand{\meets}{^*\!\!\leftarrow \cdot \rightarrow^* }


\newcommand{\mCP}[1]{\mathsf{CP}(#1)}		% Critical Pair
\newcommand{\mCPR}{\mCP{\mcR}}		% CP(R)

\newcommand{\MUL}[2]	% multiplication
{\mf(#1,#2)}			% mul(x,y)
%{#1\cdot #2}			% x.y

\newcommand{\ADD}[2]	% addition
{\add(#1,#2)}			% add(x,y)
%{#1+#2}				% x+y

\newcommand{\MYPOS}[1]{{\tt #1}}
\newcommand{\overlap}[3]{\langle #1,\MYPOS{#2}, #3 \rangle}
\newcommand{\overlapN}[4]{{_{\overlap{#1}{#2}{#3}}}^{#4:\;}}

%\newcommand{\GTKBO}{>_{\tt kbo}}
\newcommand{\GTKBOW}[2]{\texttt{SMT}(#1\!>_\texttt{kbo}\!#2)}
\newcommand{\GTKBOP}[2]{\texttt{SMT}(#1\!>_\texttt{kbo}'\!#2)}

\newcommand{\UPL}{\infer
	[(\sigma)
		\quad\sigma=\mgu(l,l'), l'\!\not\in\mcV, l\sigma\rho\sgr r\sigma\rho
	]
	{L[r]\sigma}
	{l=r & L[l']}	
}

\newcommand{\emptyclause}{\square}

% !TEX root = m3talk.tex
% !TEX encoding = UTF-8 Unicode

%% ==============================================================
%% ### BEAMER ###

\usetheme[]{CambridgeUS}	
\usecolortheme{seagull}			% seahorse, fly, dolphin, dove, beetle, seagull ...
\useinnertheme{circles}			% tree, smoothtree, infolines, smoothbars	
\usefonttheme{professionalfonts} 	% professionalfonts serif structurebold structureitalicserif structuresmallcapsserif
	
\beamertemplatenavigationsymbolsempty

%% ==============================================================
%% ### "PACKAGES" ###

% BY BEAMER: xcolor, amsmath, amsthm, calc, geometry, hyperref, extsizes

\usepackage[utf8x]{inputenc} 								% Eingabekodierung	
\usepackage[polutonikogreek,ngerman,english]{babel}	
\usepackage[T1]{fontenc}	 								% Ausgabekodierung (PDF)
%\usepackage[usenames,dvipsnames,svgnames,table]{xcolor}		% BEAMER, Farben
%\usepackage{amsmath}									% BEAMTER
%\usepackage{amsthm}									% BEAMTER
%\usepackage{amssymb}									% 

\usepackage{proof} 				%\infer, \deduce
%\usepackage{marvosym}			% \Lightning

%\usepackage{soul}			% \so\caps\ul\st\hl
%\usepackage[normalem]{ulem} % \uline\uuline\sout\xout\uwave
%\usepackage{transparent}
%\usepackage{listings}
%\usepackage{multirow}
%\usepackage{pdfpages}
%\usepackage[weather]{ifsym} % does not work with tex writer
%\let\Sun\relax % defined in marvosym too
%\let\Lightning\relax % defined in marvosym too

%\usepackage{ulsy}	% \blitza \blitzb ... \blitze (does not exist anymore? just not included?)

%% ==============================================================
%% ### MY COLOR DEFINITONS ###	

%\colorlet{col:a}{Fuchsia}
%\colorlet{col:b}{Blue}
\colorlet{col:g}{Gray}
%\colorlet{col:o}{Orange}
\colorlet{col:hi}{Green}
\colorlet{col:lo}{Red}
\colorlet{col:na}{Gray}
\colorlet{col:n}{Blue}
%\colorlet{col:max}{Violet}
%\colorlet{col:smx}{RoyalBlue}

%\newcommand{\colA}{\color{col:a}}	% example a
%\newcommand{\colB}{\color{col:b}}	% example b
\newcommand{\colG}{\color{col:g}}	% neutral
%\newcommand{\colO}{\color{col:o}}	% old
\newcommand{\colN}{\color{col:n}}	% new
\newcommand{\colHI}{\color{col:hi}}	% hi, true
\newcommand{\colLO}{\color{col:lo}}	% lo, false
\newcommand{\colNA}{\color{col:na}}	% not available		
%\newcommand{\colMAX}{\color{col:max}}	% maximal
%\newcommand{\colSMX}{\color{col:smx}}	% strictly maximal		

%\newcommand{\MYa}{\color{Fuchsia}}
%\newcommand{\MYb}{\color{Blue}}
%\newcommand{\MYg}{\color{Gray}}
%\newcommand{\MYo}{\color{Orange}}
%
%\newcommand{\MYhi}{\color{Green}}
%\newcommand{\MYlo}{\color{Red}}
%\newcommand{\MYna}{\color{Gray}}
%
%\newcommand{\MYA}[1]{{\MYa#1}}
%\newcommand{\MYB}[1]{{\MYb#1}}
%\newcommand{\MYG}[1]{{\MYg#1}}
%\newcommand{\MYO}[1]{{\MYo#1}}
%
%\newcommand{\MYHI}[1]{{\MYhi#1}}
%\newcommand{\MYLO}[1]{{\MYlo#1}}
%\newcommand{\MYNA}[1]{{\MYna#1}}
%
%\newcommand{\MYS}[1]{{\color{Violet}#1}}
\newcommand{\MYM}[1]{{\color{RoyalBlue}#1}}
%
%\newcommand{\mLightning}{{\text{\Lightning}}}

%% ==============================================================

%% ==============================================================
%% ### MISC ###

\newcommand{\EQ}{\simeq}
\newcommand{\NEQ}{\not\simeq}
\newcommand{\foEQ}{\approx}		%	{\simeq}
\newcommand{\foNEQ}{\not\foEQ}

%\newcommand\mytop[2]{\genfrac{}{}{0pt}{}{#1}{#2}}
%\newcommand{\mygreek}[1]{\selectlanguage{polutonikogreek}#1\selectlanguage{english}}
%\newcommand{\mygreek}[1]{{\selectlanguage{polutonikogreek}#1}\selectlanguage{english}}
%\renewcommand{\mygreek}[1]{\foreignlanguage{polutonikogreek}{#1}}

%\newcommand{\iSUB}[2]{#2\!\mapsto\!#1}
%\newcommand{\BgSyntaxTree}{\usebackgroundtemplate{\transparent{0.1}\includegraphics[width=\paperwidth]{SyntaxTreeBackground.png}}}

\newcommand{\MYEM}[1]{\emph{\MYM{#1}}}

%% ==============================================================
%% ### MY MATH ENVIRONMENTS ###

\theoremstyle{plain}
%\newtheorem{theorem}{Theorem}
%\newtheorem{proposition}{Proposition}
%\newtheorem{lemma}{Lemma}
%\newtheorem*{corollary}{Corollary}

\theoremstyle{definition}
%\newtheorem{definition}{Definition}
\newtheorem{Conjecture}{Conjecture}
%\newtheorem*{example}{Example}
%\newtheorem{algorithm}{Algorithm}
\newtheorem{procedure}{Procedure}

\theoremstyle{remark}
\newtheorem*{remark}{Remark}
%\newtheorem*{note}{Note}
%\newtheorem{case}{Case}

%% ==============================================================
%% ### MY MATH DEFINITIONS ###

% math operators

\DeclareMathOperator{\arity}{arity}
\DeclareMathOperator{\var}{var}
\DeclareMathOperator{\pos}{pos}
\DeclareMathOperator{\T}{T}
\DeclareMathOperator{\dom}{dom}
\DeclareMathOperator{\rng}{rng}
\DeclareMathOperator{\img}{img}
\DeclareMathOperator{\mgu}{mgu}
\DeclareMathOperator{\wgt}{W\!}
\DeclareMathOperator{\sel}{sel}
\DeclareMathOperator{\mul}{mul}
\DeclareMathOperator{\add}{add}

\DeclareMathOperator{\unif}{unifiable}
\DeclareMathOperator{\inst}{instance}
\DeclareMathOperator{\gene}{generalization}
\DeclareMathOperator{\vari}{variant}

% constant (function, predicate) symbols

\newcommand{\mf}{{\mathsf f}}
\newcommand{\mg}{{\mathsf g}}
\newcommand{\mh}{{\mathsf h}}
\newcommand{\ma}{{\mathsf a}}
\newcommand{\mb}{{\mathsf b}}
\newcommand{\mc}{{\mathsf c}}
\newcommand{\md}{{\mathsf d}}
\newcommand{\mx}{{\mathsf x}}
\newcommand{\my}{{\mathsf y}}
\newcommand{\msucc}{{\mathsf s}}
\newcommand{\mpred}{{\mathsf p}}
\newcommand{\mA}{{\mathsf A}}
\newcommand{\mB}{{\mathsf B}}
\newcommand{\mP}{{\mathsf P}}
\newcommand{\mQ}{{\mathsf Q}}

% caligraphic symbols

\newcommand{\mcC}{{\mathcal C}}
\newcommand{\mcD}{{\mathcal D}}
\newcommand{\mcE}{{\mathcal E}}
\newcommand{\mcF}{{\mathcal F}}
\newcommand{\mcM}{{\mathcal M}}
\newcommand{\mcR}{{\mathcal R}}
\newcommand{\mcT}{{\mathcal T}}
\newcommand{\mcV}{{\mathcal V}}

\newcommand{\VAR}{\mcV\mathtt{ar}}
% fraktal symbols

\newcommand{\mfC}{{\mathfrak C}}
\newcommand{\mfL}{{\mathfrak L}}
\newcommand{\mfR}{{\mathfrak R}}
\newcommand{\mfT}{{\mathfrak T}}

% tt symbols

\newcommand{\mtS}{{\mathtt S}}
\newcommand{\sgr}{\succ_{\mathsf gr}}

% 

\newcommand{\joins}{\rightarrow^*\cdot^*\!\!\leftarrow}
\newcommand{\meets}{^*\!\!\leftarrow \cdot \rightarrow^* }


\newcommand{\mCP}[1]{\mathsf{CP}(#1)}		% Critical Pair
\newcommand{\mCPR}{\mCP{\mcR}}		% CP(R)

\newcommand{\MUL}[2]	% multiplication
{\mf(#1,#2)}			% mul(x,y)
%{#1\cdot #2}			% x.y

\newcommand{\ADD}[2]	% addition
{\add(#1,#2)}			% add(x,y)
%{#1+#2}				% x+y

\newcommand{\MYPOS}[1]{{\tt #1}}
\newcommand{\overlap}[3]{\langle #1,\MYPOS{#2}, #3 \rangle}
\newcommand{\overlapN}[4]{{_{\overlap{#1}{#2}{#3}}}^{#4:\;}}

%\newcommand{\GTKBO}{>_{\tt kbo}}
\newcommand{\GTKBOW}[2]{\texttt{SMT}(#1\!>_\texttt{kbo}\!#2)}
\newcommand{\GTKBOP}[2]{\texttt{SMT}(#1\!>_\texttt{kbo}'\!#2)}

\newcommand{\UPL}{\infer
	[(\sigma)
		\quad\sigma=\mgu(l,l'), l'\!\not\in\mcV, l\sigma\rho\sgr r\sigma\rho
	]
	{L[r]\sigma}
	{l=r & L[l']}	
}

\newcommand{\emptyclause}{\square}

%% ==============================================================
%% ### TIKZ ###

\usepackage{tikz}
\usetikzlibrary{automata,arrows,shapes
%arrows,shapes,snakes,automata,backgrounds,petri, positioning
%,,automata,shadows,fit,
% decorations,pathmorphing
% graphs
}

\tikzset{
	->,
%	>=stealth', 
%	shorten >=1pt, 
%	auto,
	node distance=2.1cm, 
%	semithick,
	minimum size=0,
%	inner sep=0,
%	outer sep=2mm,
%
%	initial text=$\varepsilon$,
%	
	every state/.style={
		fill=red,
%		draw=none,
		text=white,
%		radius=0.5cm
	},
	my/.style={ rectangle, draw=red	},
%	sloped,below
}


%\tikzstyle{myrect} = [rectangle,draw=black,rounded corners, minimum height=3em, thick, text centered,text width=5.5em]
%\tikzstyle{mykaro} = [diamond,draw=black,rounded corners, thick, text centered,text width=4em]
%\tikzstyle{mycircle} = [circle,draw=black,thick, text centered, minimum height=3.5em,text width=4em,text width=3.5em]
%\tikzstyle{myarrow} = [thick,->,>=stealth]
%
%\tikzstyle{myframe} = [rectangle,draw=black,rounded corners, minimum height=3em, thick, text centered,text width=5.5em]

%% ==============================================================

%% ==============================================================

% !TEX encoding = UTF-8 Unicode

\documentclass[ 
% amsmath=
% amsthm=
% array=
% calc
% enumerate=
% extsizes
% geometry
% hyperref
xcolor={usenames,dvipsnames,svgnames,tablem} 
%,bigger
,handout
]{beamer}
% automatically loaded: xcolor, amsmath, amsthm, calc, geometry, hyperref, extsizes

% !TeX root = ../m3Handout.tex
% !TeX encoding = UTF-8
% !TeX spellcheck = en_US

% !TeX root = ../m3Headout.tex
% !TeX encoding = UTF-8
% !TeX spellcheck = en_US

%% ==============================================================
%% ### BEAMER ###

\usetheme[]{CambridgeUS}	
\usecolortheme{seagull}			% seahorse, fly, dolphin, dove, beetle, seagull ...
\useinnertheme{circles}			% tree, smoothtree, infolines, smoothbars	
\usefonttheme{professionalfonts} 	% professionalfonts serif structurebold structureitalicserif structuresmallcapsserif
	
\beamertemplatenavigationsymbolsempty

%% ==============================================================
%% ### "PACKAGES" ###

% BY BEAMER: xcolor, amsmath, amsthm, calc, geometry, hyperref, extsizes

\usepackage{calc}

\usepackage[utf8x]{inputenc} 								% Eingabekodierung	
\usepackage[english]{babel}	
\usepackage[T1]{fontenc}	 								% Ausgabekodierung (PDF)
%\usepackage[usenames,dvipsnames,svgnames,table]{xcolor}		% BEAMER, Farben
%\usepackage{amsmath}									% BEAMTER
%\usepackage{amsthm}									% BEAMTER
\usepackage{amssymb}									% \ngtpreq

\usepackage{proof} 				%\infer, \deduce
%\usepackage{marvosym}			% \Lightning

%\usepackage{soul}			% \so\caps\ul\st\hl
%\usepackage[normalem]{ulem} % \uline\uuline\sout\xout\uwave
%\usepackage{transparent}
%\usepackage{listings}
%\usepackage{multirow}
%\usepackage{pdfpages}
%\usepackage[weather]{ifsym} % does not work with tex writer
%\let\Sun\relax % defined in marvosym too
%\let\Lightning\relax % defined in marvosym too

%\usepackage{ulsy}	% \blitza \blitzb ... \blitze (does not exist anymore? just not included?)

\usepackage{tabularx}

% !TeX root = ../m3Handout.tex
% !TeX encoding = UTF-8
% !TeX spellcheck = en_US

%% color definitions

%\colorlet{col:a}{Fuchsia}
%\colorlet{col:b}{Blue}
\colorlet{colG}{Gray}
\colorlet{colO}{Orange}
\colorlet{colHi}{Green}
\colorlet{colLo}{Red}
\colorlet{colNa}{Gray}
\colorlet{colN}{Blue}
\colorlet{colEm}{RoyalBlue}
\colorlet{colMax}{Violet}
\colorlet{colSmx}{RoyalBlue}

%\newcommand{\colA}{\color{col:a}}	% example a
%\newcommand{\colB}{\color{col:b}}	% example b
\newcommand{\colG}{\color{colG}}	% neutral
%\newcommand{\colO}{\color{col:o}}	% old
\newcommand{\colN}{\color{colN}}	% new
\newcommand{\colHi}{\color{colHi}}	% hi, true
\newcommand{\colLo}{\color{colLo}}	% lo, false
\newcommand{\colNA}{\color{colNa}}	% not available		
\newcommand{\colMax}{\color{colMax}}	% maximal
\newcommand{\colSmx}{\color{colSmx}}	% strictly maximal		

%\newcommand{\MYa}{\color{Fuchsia}}
%\newcommand{\MYb}{\color{Blue}}
%\newcommand{\MYg}{\color{Gray}}
%\newcommand{\MYo}{\color{Orange}}
%
%\newcommand{\MYhi}{\color{Green}}
%\newcommand{\MYlo}{\color{Red}}
%\newcommand{\MYna}{\color{Gray}}
%
%\newcommand{\MYA}[1]{{\MYa#1}}
%\newcommand{\MYB}[1]{{\MYb#1}}
%\newcommand{\MYG}[1]{{\MYg#1}}
%\newcommand{\MYO}[1]{{\MYo#1}}
%
%\newcommand{\MYHI}[1]{{\MYhi#1}}
%\newcommand{\MYLO}[1]{{\MYlo#1}}
%\newcommand{\MYNA}[1]{{\MYna#1}}
%
%\newcommand{\MYS}[1]{{\color{Violet}#1}}
%\newcommand{\MYM}[1]{{\color{RoyalBlue}#1}}
%
%\newcommand{\mLightning}{{\text{\Lightning}}}

\newcommand{\BEGINX}{\begin{example}}
\newcommand{\ENDX}{\end{example}}

\newcommand{\BEGIN}{\begin{exampleblock}}
\newcommand{\END}{\end{exampleblock}}

\newcommand{\FOL}{$_\mathtt{FOL}$}
% !TeX root = ../m3Handout.tex
% !TeX encoding = UTF-8
% !TeX spellcheck = en_US

%% ==============================================================
%% ### MISC ###

%\newcommand{\EQ}{\simeq}
%\newcommand{\NEQ}{\not\simeq}
\newcommand{\foEQ}{\approx}		%	{\simeq}
\newcommand{\foNEQ}{\not\foEQ}

\newcommand\TOP[2]{\genfrac{}{}{0pt}{}{#1}{#2}}
\newcommand\TOPTEXT[2]{\TOP{\text{#1}}{\text{#2}}}
%\newcommand{\mygreek}[1]{\selectlanguage{polutonikogreek}#1\selectlanguage{english}}
%\newcommand{\mygreek}[1]{{\selectlanguage{polutonikogreek}#1}\selectlanguage{english}}
%\renewcommand{\mygreek}[1]{\foreignlanguage{polutonikogreek}{#1}}

%\newcommand{\iSUB}[2]{#2\!\mapsto\!#1}
%\newcommand{\BgSyntaxTree}{\usebackgroundtemplate{\transparent{0.1}\includegraphics[width=\paperwidth]{SyntaxTreeBackground.png}}}

\newcommand{\EMPH}[1]{\emph{\textcolor{colEm}{#1}}}

%% ==============================================================
%% ### MY MATH ENVIRONMENTS ###

\theoremstyle{plain}
%\newtheorem{theorem}{Theorem}
%\newtheorem{proposition}{Proposition}
%\newtheorem{lemma}{Lemma}
%\newtheorem*{corollary}{Corollary}

\theoremstyle{definition}
%\newtheorem{definition}{Definition}
\newtheorem{Conjecture}{Conjecture}
%\newtheorem*{example}{Example}
%\newtheorem{algorithm}{Algorithm}
\newtheorem{procedure}{Procedure}
\newtheorem{goal}{Goal}
\newtheorem{notation}{Notation}

\theoremstyle{remark}
%\newtheorem*{remark}{Remark}
\newtheorem*{observation}{Observation}
%\newtheorem*{note}{Note}
%\newtheorem{case}{Case}

%% ==============================================================
%% ### MY MATH DEFINITIONS ###

% math operators

%\DeclareMathOperator{\arity}{arity}
%\DeclareMathOperator{\var}{var}
%\DeclareMathOperator{\pos}{pos}
%\DeclareMathOperator{\T}{T}
%\DeclareMathOperator{\dom}{dom}
%\DeclareMathOperator{\rng}{rng}
%\DeclareMathOperator{\img}{img}
\DeclareMathOperator{\mgu}{mgu}	% most general unifier
%\DeclareMathOperator{\wgt}{W\!}
%\DeclareMathOperator{\sel}{sel}
%\DeclareMathOperator{\mul}{mul}
%\DeclareMathOperator{\add}{add}

\DeclareMathOperator{\UNIF}{unifiable}
\DeclareMathOperator{\INST}{instance}
\DeclareMathOperator{\GNRL}{generalization}
\DeclareMathOperator{\VRNT}{variant}
\DeclareMathOperator{\PSTR}{pstr}

\newcommand{\NGTPREQ}{\not\succcurlyeq}

% constant (function, predicate) symbols

\newcommand{\mf}{{\mathsf f}}
\newcommand{\mg}{{\mathsf g}}
\newcommand{\mh}{{\mathsf h}}
\newcommand{\ma}{{\mathsf a}}
\newcommand{\mb}{{\mathsf b}}
\newcommand{\mc}{{\mathsf c}}
\newcommand{\md}{{\mathsf d}}
\newcommand{\mx}{{\mathsf x}}
\newcommand{\my}{{\mathsf y}}
\newcommand{\msucc}{{\mathsf s}}
\newcommand{\mpred}{{\mathsf p}}
\newcommand{\mA}{{\mathsf A}}
\newcommand{\mB}{{\mathsf B}}
\newcommand{\mP}{{\mathsf P}}
\newcommand{\mQ}{{\mathsf Q}}

% caligraphic symbols

\newcommand{\mcC}{{\mathcal C}}
\newcommand{\mcD}{{\mathcal D}}
%\newcommand{\mcE}{{\mathcal E}}
%\newcommand{\mcF}{{\mathcal F}}
%\newcommand{\mcM}{{\mathcal M}}
%\newcommand{\mcO}{{\mathcal O}}
\newcommand{\mcP}{{\mathcal P}}
%\newcommand{\mcR}{{\mathcal R}}
%\newcommand{\mcT}{{\mathcal T}}
\newcommand{\mcV}{{\mathcal V}}

\newcommand{\Var}{{}\mcV\mathsf{ar}}
\newcommand{\Dom}{{}\mcD\mathsf{om}}
\newcommand{\Pos}{{}\mcP\mathsf{os}}
\newcommand{\PosStr}{\Pos^\Sigma}

% fraktal symbols

\newcommand{\mfC}{{\mathfrak C}}
\newcommand{\mfL}{{\mathfrak L}}
\newcommand{\mfR}{{\mathfrak R}}
\newcommand{\mfT}{{\mathfrak T}}

\newcommand{\SIGA}{{\mathcal A}}
\newcommand{\SIGC}{{\mathcal C}}
\newcommand{\SIGE}{{\mathcal E}}
\newcommand{\SIGF}{{\mathcal F}}
\newcommand{\SIGL}{{\mathcal L}}
\newcommand{\SIGP}{{\mathcal P}}
\newcommand{\SIGS}{{\mathcal S}}
\newcommand{\SIGT}{{\mathcal T}}
\newcommand{\SIGV}{{\mathcal V}}
% tt symbols

\newcommand{\mtS}{{\mathtt S}}
\newcommand{\sgr}{\succ_{\mathsf gr}}

% 

\newcommand{\TI}[1]{^{^{#1:}}\!}
\newcommand{\ANGLES}[1]{\langle#1\rangle}

\newcommand{\joins}{\rightarrow^*\cdot^*\!\!\leftarrow}
\newcommand{\meets}{^*\!\!\leftarrow \cdot \rightarrow^* }


\newcommand{\mCP}[1]{\mathsf{CP}(#1)}		% Critical Pair
\newcommand{\mCPR}{\mCP{\mcR}}		% CP(R)

\newcommand{\MUL}[2]	% multiplication
{\mf(#1,#2)}			% mul(x,y)
%{#1\cdot #2}			% x.y

\newcommand{\ADD}[2]	% addition
{\add(#1,#2)}			% add(x,y)
%{#1+#2}				% x+y

\newcommand{\MYPOS}[1]{{\tt #1}}
\newcommand{\overlap}[3]{\langle #1,\MYPOS{#2}, #3 \rangle}
\newcommand{\overlapN}[4]{{_{\overlap{#1}{#2}{#3}}}^{#4:\;}}

%\newcommand{\GTKBO}{>_{\tt kbo}}
\newcommand{\GTKBOW}[2]{\texttt{SMT}(#1\!>_\texttt{kbo}\!#2)}
\newcommand{\GTKBOP}[2]{\texttt{SMT}(#1\!>_\texttt{kbo}'\!#2)}

\newcommand{\UPL}{\infer
	[(\sigma)
		\quad\sigma=\mgu(l,l'), l'\!\not\in\mcV, l\sigma\rho\sgr r\sigma\rho
	]
	{L[r]\sigma}
	{l=r & L[l']}	
}

\newcommand{\emptyclause}{\square}

% !TeX root = ../m3Handout.tex
% !TeX encoding = UTF-8
% !TeX spellcheck = en_US

%% ### TIKZ ###

\usepackage{tikz}
\usetikzlibrary{automata,arrows,shapes,trees,decorations.pathmorphing
% snakes have been supersede by decorations
%arrows,shapes,,automata,backgrounds,petri, positioning
%,,automata,shadows,fit,
% decorations,pathmorphing
% graphs
}

\tikzset{
%	->,
%	>=stealth', 
%	shorten >=1pt, 
%	auto,
	node distance=2.1cm, 
%	semithick,
	minimum size=0,
	inner sep=1,
	outer sep=1mm,
%
	initial text=$\varepsilon$,
%	
	every state/.style={
		fill=red,
%		draw=none,
%		text=white,
		radius=0.1em
	},
	my/.style={ rectangle, draw=red	},
%	sloped,below
}


%\tikzstyle{myrect} = [rectangle,draw=black,rounded corners, minimum height=3em, thick, text centered,text width=5.5em]
%\tikzstyle{mykaro} = [diamond,draw=black,rounded corners, thick, text centered,text width=4em]
%\tikzstyle{mycircle} = [circle,draw=black,thick, text centered, minimum height=3.5em,text width=4em,text width=3.5em]
%\tikzstyle{myarrow} = [thick,->,>=stealth]
%
%\tikzstyle{myframe} = [rectangle,draw=black,rounded corners, minimum height=3em, thick, text centered,text width=5.5em]

%% ==============================================================

\newcommand{\ORIGIN}{
	\node[orange](ORIGIN) at (0,0) {\scriptsize$\odot$};
	\node[orange](XONE) at (1,0) {\scriptsize$\times$};
	}
%% ==============================================================

% clear command for final version

\renewcommand{\ORIGIN}{}
\providecommand{\PAUSE}{}






% ********************************

\author{Alexander Maringele}
\title{Term-Indexing
%\\and\\First Order Theorem Proving
}
\subtitle{for Instantiation-Based\\First Order Theorem Proving}
\date{January 27th, 2016}
%======

\begin{document}

\selectlanguage{english}

%\Cloud

% === TITLE ===
\frame[<+->]{
\maketitle
}

% === REFERENCES ===
\section*{References}
\frame[<+->]{
\frametitle{References}

\nocite{SRV2001ti,RV2003eir}	% ZHM2009jar
\bibliographystyle{amsalpha}
\bibliography{biblio}

}

% === OVERVIEW ===

\section*{Outline}
\setcounter{tocdepth}{1}
\frame[<+->]{
\frametitle{Outline}
	\tableofcontents
}



%====================================================================
% BEGIN: CONTENT ----------------------------------------------------
%====================================================================

%\input{xy}

\section{Motivation}

%\subsection{Notation}
%\begin{frame}
%% Notation First Order Clause Logic
\begin{notation}[First Order Logic]
\begin{enumerate}
\item $\Sigma = (\SIGV, \SIGF, \SIGP)$ \hfill signature
%\item $P \in \Sigma_\mP = \{ \mA, \mB, \ldots \}$ \hfill predicate symbols 
%\item $F \in \Sigma_\mf = \{ \mc, \mf,\mg,\ldots \}$ \hfill (constant) function symbols
%\item $V = \{ x,y,z, \ldots\} $ \hfill variables
%\item $C ::=  \lnot \mid \lor \mid {\colG \land \mid\ \rightarrow}$\hfill connectives
%\item {\colG $Q ::= \forall, \exists$ \hfill quantifiers}
\item $\SIGT = \SIGV \cup \{ f(t_1,\ldots,t_n)\mid f\in\SIGF_{(n)}, t_i \in \SIGT \}$\hfill terms
\item $\SIGA = \{ P ( t_1, \ldots, t_n )\mid P\in\SIGP_{(n)}, t_i \in \SIGT\}
\cup
\{ s \foEQ t \mid s,t\in \SIGT\}$\hfill  atoms
%\item {\colG $F ::= A \mid (\lnot F) 
%	\mid (F \lor F)  \mid (F \land F)  
%	\mid (F \rightarrow F)
%	\mid (\forall F)
%	\mid (\exists F) 
%	$ 
%	\hfill formulas
%	
%	}
\item $\SIGL = \SIGA \cup \{ \lnot A \mid A\in \SIGA \}$\hfill literals
\item $\SIGC = 2^\SIGL$, e.g. $\mf(x)\foEQ\ma \lor \ma\foNEQ\mb$\hfill clauses
\item $\SIGS = 2^\SIGC$, e.g. $\{ \mf(x)\foEQ\ma \lor \ma\foNEQ\mb, \mf(x)\foEQ\mb\}$\hfill clause sets
\end{enumerate}
%A set of clauses is equivalent to a conjunction of universally quantified disjunctions of literals
%and equivalent to a universally quantified conjunction of variable distinct disjunctions of literals.

%\item $S ::= \{ C,\lots,C \}$\hfill sets of clauses
%\item Formulae $F::=$



\end{notation}

\[
\{ C_1, \ldots, C_n \} \equiv 
\forall\Var(C_1) C_1 
\land \ldots \land
\forall\Var(C_1) C_n 
\]

%\end{frame}

% === saturation-based theorem provers ===
\subsection{Reductio ad absurdum}
%\begin{frame}\frametitle{Resolution}
%	\begin{procedure}	
	\begin{enumerate}
	\item Transform the negation of a conjecture $F$
	
	into a equisatisfiable set of clauses $S$.
	\item Expand $S$ with derived clauses 
	in a sufficient$^{\circ\diamond\star}$ way.
	\item Stop if 
	\begin{enumerate}
	\item a proof for unsatisfiability of $S$ has been found\hfill{\colHi F is a theorem}
	\item {\colG the set $S$ is saturated, hence satisfiable}\hfill{\colG F is not a theorem} 
	\item time's up or space's out\hfill \colG we don't know
	\end{enumerate}
	Otherwise, continue with 2.
	\end{enumerate}
	\end{procedure}
	

	
%The resolution rule can be traced back to Davis and Putnam (1960);[1] however, their algorithm required to try all ground instances of the given formula. This source of combinatorial explosion was eliminated in 1965 by John Alan Robinson's syntactical unification algorithm, which allowed one to instantiate the formula during the proof "on demand" just as far as needed to keep refutation completeness.[2]
%	\begin{goal}[]
	A sound$^\circ$, refutational complete$^\diamond$, and effective$^\star$ procedure.
	\end{goal}
%\end{frame}

% === the big picture === 
\begin{frame}
\frametitle{Resolution}
\input{BLOCKS/TikzResolution}
	\begin{goal}[]
	A sound$^\circ$, refutational complete$^\diamond$, and effective$^\star$ procedure.
	\end{goal}
\end{frame}

% === Example Superpostition ===
%\subsection{Superposition}
%\begin{frame}
%\input{BLOCKS/ExampleSuperposition}
%\end{frame}

% === Example Inst-Gen-Eq ===
%\subsection{Inst-Gen-Eq}
%\begin{frame}
%\input{BLOCKS/ExampleInstGenEq}
%\end{frame}

%\begin{remark}
%	Superposition [Bachmair, Ganzinger],
%	InstGenEq (Ganzinger, Korovin) are sound and refutational complete with a \EMPH{fair} strategy.
%	\end{remark}

% === example infinite domain ===
%\begin{frame}
%% !TEX root = ../m3talk.tex
% !TEX encoding = UTF-8 Unicode

\begin{align*}
S_1 &= \{ 
	\mf(x,x)\foNEQ a,
	\mf(y,\mg(y)) \foEQ a,\\
	&\qquad\mf(x,y)\foNEQ a \lor \mf (y,z)\foNEQ a \lor \mf(x,z)\foEQ a 
	\}
	\\
%S_1\bot &= \mf(\bot,\bot)\foNEQ a,
%	\mf(\bot,\mg(\bot)) \foEQ a,
%	\mf(\bot,\bot)\foNEQ a \lor \mf (\bot,\bot)\foNEQ a \lor f(\bot,\bot)\foEQ a \\
\end{align*}
%\end{frame}

\begin{frame}

	\begin{goal}[]
	A sound$^\circ$, refutational complete$^\diamond$, and effective$^\star$ procedure.
	\end{goal}

\end{frame}

% === Term retrieval problems ===

\frame{
\frametitle{Term retrieval problems}
\begin{itemize}
\item Find terms that are variants of a given term.
$\vari(\ell,t) \Leftrightarrow \exists\sigma\ \ell\sigma = t$ and $\sigma$ is renaming.
\item Find terms that are unifiable with a given term. 
$\unif(\ell,t) \Leftrightarrow \exists\sigma\ \ell\sigma = t\sigma$
\item Find terms that are instances of a given term.
$\inst(\ell,t) \Leftrightarrow \exists\sigma\ \ell = t\sigma$
\item Find terms that are generalizations of a given term.
$\gene(\ell,t) \Leftrightarrow \exists\sigma\ \ell\sigma = t$
\end{itemize}
\begin{definition}
\end{definition}
}

% === Definition Position ===

\subsection{Position}
\begin{frame}
% p.1860, 2.1 Definition
% 703811, p.26, 2.1.14
\begin{definition}
A position is a sequence of positive integers.
The empty sequence $\varepsilon$ denotes the root position,
$pq$ denotes the concatenation of positions. 
%$p$ and $q$,
%The set of positions is recursively defined: 
%\[
%\pos(t)=\left\{ 
%\begin{array}{ll}
%	\{ \varepsilon \} & \text{if }t\text{ is variable}\\
%	\{ \varepsilon \} \cup \{ ip\mid 1\leq i\leq n \land p \in \pos(t_i) \} & \text{if }t=\mf(t_1,\ldots,t_n)
%\end{array}
%\right.
%\]
%The subterm of $t$ at position $p\in\pos(t)$ is defined:
%\[
%t|_p = \left\{
%\begin{array}{ll}
%	t & \text{if }p=\varepsilon\\
%	t_i|_q & \text{if }t=\mf(t_1,\ldots,t_n) \text{ and } p=iq
%\end{array}
%\right.
%\]
$\Pos(t)$ denotes the set of positions in term $t$, and $t|_p$ denotes the subterm of $t$ at positon $p\in\Pos(t)$.
\end{definition}

\begin{definition}
A postion string is a nonempty string of the form $\langle p_1,s_1\rangle\ldots\langle p_n,s_n\rangle$
where $p_i$ are positions and $s_i$ are function or variable symbols and
\begin{enumerate}
\item
if $p_i$ is a proper prefix of $p_j$ then $i<j$
\item
\end{enumerate}
\end{definition}
\end{frame}

% === ====

\begin{frame}
% 4.1 Position strings, 4.1 Defintion
\begin{definition}

\end{definition}
\end{frame}

% === state machine ====

%\begin{frame}
%\begin{tikzpicture}[->,>=stealth',shorten >=1pt,auto,node distance=2.1cm,
%                    semithick]
%  \tikzstyle{every state}=[fill=red,draw=none,text=white]
%
%  \node[initial,state] (A)                    {$q_a$};
%  \node[state]         (B) [above right of=A] {$q_b$};
%  \node[state]         (D) [below right of=A] {$q_d$};
%  \node[state]         (C) [below right of=B] {$q_c$};
%  \node[state]         (E) [below of=D]       {$q_e$};
%
%  \path (A) edge              node {0,1,L} (B)
%            edge              node {1,1,R} (C)
%        (B) edge [loop above] node {1,1,L} (B)
%            edge              node {0,1,L} (C)
%        (C) edge              node {0,1,L} (D)
%            edge [bend left]  node {1,0,R} (E)
%        (D) edge [loop below] node {1,1,R} (D)
%            edge              node {0,1,R} (A)
%        (E) edge [bend left]  node {1,0,R} (A);
%\end{tikzpicture}
%
%\href{http://www.texample.net/tikz/examples/state-machine/}{www.texample.net/tikz/examples/state-machine/}
%\end{frame}
\section{path indexing}

\begin{frame}

\begin{gather*}
\{
	\TI{1}\mh(\mf(x,x)),
	\TI{2}\mh(\mg(\ma,x)), 
	\TI{3}\mh(\mf(y,z))
	\TI{4}\mh(\mg(\ma,y)), 
	\TI{5}\mh(\mf(y,x)), 
	\TI{6}\mh(\mg(y,a))
\}
\end{gather*}
%
\begin{center}
\begin{tikzpicture}[->,sloped,above]

\node (root) at (0,2.5) {.};

\node (h) at (1,2.5) {.};

\node (h1) at (2,2.5) {.};

\node (h1f) at (3,4.5) {.};
\node (h1g) at (3,1.5) {.};

\node (h1f1) at (4,5) {.};
\node (h1f2) at (4,4) {.};
\node (h1g1) at (4,2.5) {.};
\node (h1g2) at (4,0.5) {.};

\node (h1f1x) at (6,5) {\{1,3,5\}};
\node (h1f2x) at (6,4) {\{1,3,5\}};
\node (h1g1a) at (6,3) {\{2,4\}};
\node (h1g1x) at (6,2) {\{6\}};
\node (h1g2a) at (6,1) {\{6\}};
\node (h1g2x) at (6,0) {\{2,4\}};

\path (root) edge node {$\mh$} (h)
	(h) edge node {$1$} (h1)
	
	(h1)
		edge node {$\mf$} (h1f)
		edge node {$\mg$} (h1g)
	
	(h1f)
		edge node {$1$} (h1f1)
		edge node {$2$} (h1f2)
	
	(h1g)
		edge node {$1$} (h1g1)
		edge node {$2$} (h1g2)

	(h1f1)
		% edge node {$\ma$} (h1f1a)
		edge node {$*$} (h1f1x)
		
	(h1f2)
%		edge node {$\ma$} (h1f2a)
		edge node {$*$} (h1f2x)
		

	(h1g1)
		edge node {$\ma$} (h1g1a)
		edge node {$*$} (h1g1x)

	(h1g2) 
		edge node {$\ma$} (h1g2a)
		edge node {$*$} (h1g2x)
;
\end{tikzpicture}
\end{center}
$\mh(\mg(y,x)) \mapsto \{ \mh.1.\mg.1.{*}, \mh.1.\mg.2.{*} \}$
\end{frame}

% === ===
\subsection{subterm indexing}
\begin{frame}

\begin{gather*}
\{
	\TI{1}\mh(\mf(x,x)),
	\TI{2}\mh(\mg(\ma,x)), 
	\TI{3}\mh(\mf(y,z))
	\TI{4}\mh(\mg(\ma,y)), 
	\TI{5}\mh(\mf(y,x)), 
	\TI{6}\mh(\mg(y,a))
\}
\end{gather*}
%
\begin{center}
\begin{tikzpicture}[->,sloped,above]
\node (root) at (0,2.5) {$\bullet$};
%% right
\node (h) at (1,2.5) {.};
%
\node (h1) at (2,2.5) {.};
%
\node (h1f) at (3,4.5) {.};
\node (h1g) at (3,1.5) {.};
%
\node (h1f1) at (4,5) {.};
\node (h1f2) at (4,4) {.};
\node (h1g1) at (4,2.5) {.};
\node (h1g2) at (4,0.5) {.};
%
\node (h1f1x) at (6,5) {$\{1,3,5\}$};
\node (h1f2x) at (6,4) {$\{1,3,5\}$};
\node (h1g1a) at (6,3) {$\{2,4\}$};
\node (h1g1x) at (6,2) {$\{6\}$};
\node (h1g2a) at (6,1) {$\{6\}$};
\node (h1g2x) at (6,0) {$\{2,4\}$};
%% down
\node (f) at (0,1.5) {.};
\node (f1) at (-1,1.3) {.};
\node (f1x) at (-1,0) {$\{1^{1},3^{1},5^{1}\}$};
\node (f2) at (1,1.3) {.};
\node (f2x) at (1,0) {$\{1^{1},3^{1},5^{1}\}$};
%% up
\node (g) at (0,3.5) {.};
\node (g1) at (-1,4) {.};
\node (g2) at (1,4) {.};
%% left
\node (a) at (-3,2.5) {$\{2^{1.1}, 4^{1.1}, 6^{1.2}\}$};

\path 
	(root) 
		edge node {$\mh$} (h)
		edge node {$\mf$} (f)
		edge node {$\mg$} (g)
		edge node {$\ma$} (a)

	(h) edge node {$1$} (h1)
	
	
	(h1)
		edge node {$\mf$} (h1f)
		edge node {$\mg$} (h1g)
	
	(h1f)
		edge node {$1$} (h1f1)
		edge node {$2$} (h1f2)
	
	(h1g)
		edge node {$1$} (h1g1)
		edge node {$2$} (h1g2)

	(h1f1)
		edge node {$*$} (h1f1x)
		
	(h1f2)
		edge node {$*$} (h1f2x)
		

	(h1g1)
		edge node {$\ma$} (h1g1a)
		edge node {$*$} (h1g1x)

	(h1g2) 
		edge node {$\ma$} (h1g2a)
		edge node {$*$} (h1g2x)
		
%
	(g) 	edge node {$1$} (g1)
		edge node {$2$} (g2)
		
	(f)	edge node {$1$} (f1)
		edge node {$2$} (f2)
		
	(f1)	edge node {$x$} (f1x)
	(f2)	edge node {$x$} (f2x)
;


\path (root) 
	
	
	
;

\end{tikzpicture}
\end{center}
\end{frame}

% === ===

\section{discrimination trees}
\subsection{non-linear terms and perfect filtering}
\begin{frame}

\begin{gather*}
\left\{\begin{array}{lll}
	\TI{1}\mh(\mf(x,x)), &
	\TI{2}\mh(\mg(\ma,x)), &
	\TI{3}\mh(\mf(y,z)), \\
	\TI{4}\mh(\mg(\ma,y)), &
	\TI{5}\mh(\mf(y,x)), &
	\TI{6}\mh(\mg(y,a))
\end{array}\right\}
%%
%		\TI{1}\mh(\mf(x,x)),
%		\TI{2}\mh(\mg(\ma,x)),
%		\TI{3}\mh(\mf(y,z)),
%		\TI{4}\mh(\mg(\ma,y)),
%		\TI{5}\mh(\mf(y,x)),
%		\TI{6}\mh(\mg(y,a))
%%		\\[0.5em]
%
%\ps(t) =
%\begin{cases}
%	* &\text{if $t=x\in\SIGV$} \\
%	\mf.\ps(t_1).\ps(t_2). \cdots.\ps(t_n) &\text{if $t=f(t_1,\ldots,t_n)$}
%	\end{cases}
%
\end{gather*}
%	
\begin{center}
% imperfect filtering	
\begin{tikzpicture}[->,sloped,above]

\node[] (root) at (0,0.5) {.};

\node[] (h) at (0,-0.5) {.};
\node[] (f) at (-1,-1)  {.};
\node[] (g) at (1,-1) {.};

\node[] (fx) at (-1,-2) {.};
\node[] (fxx) at (-1,-3.3) {$\{ 1,3,5 \}$ };

\node[] (ga) at (1.6,-2)  {.};
\node[] (gax) at (1.6,-3.3) {$\{ 2,4 \}$};

\node[] (gx) at (.4,-2) {.};
\node[] (gxa) at (.4,-3.3) {$\{ 6 \}$};

\path (root) edge node {$\mh$} (h)

	(h) edge node {$\mf$} (f) 
	edge node {$\mg$}(g)
	
	(f) edge node {$*$} (fx)
	(fx) edge node {$*$} (fxx)
	
	(g) edge node {$\ma$} (ga)
	edge node {$*$} (gx)
		
	(ga) edge node {$*$} (gax)
	(gx) edge node {$\ma$} (gxa)
	
	;
\end{tikzpicture}
%
\hspace{3em}
%
% perfect filtering
\begin{tikzpicture}[->,sloped,above,node distance=1cm]

\node[] (root) at (0,0.5) {.};

\node[] (h) at (0,-0.5) {.};
\node[] (f) at (-1,-1)  {.};
\node[] (g) at (1,-1) {.};

\node[] (fx) at (-1,-2) {.};
\node[] (fxy) at (-1.6,-3.3) {$\{ 3,5 \}$ };
\node[] (fxx) at (-0.4,-3.3) {$\{ 1 \}$ };

\node[] (ga) at (1.6,-2)  {.};
\node[] (gax) at (1.6,-3.3) {$\{ 2,4 \}$};

\node[] (gx) at (.4,-2) {.};
\node[] (gxa) at (.4,-3.3) {$\{ 6 \}$};

\path (root) edge node {$\mh$} (h)

	(h) edge node {$\mf$} (f) 
	edge node {$\mg$}(g)
	
	(f) edge node {$*_1$} (fx)
	(fx) edge node {$*_1$} (fxx)
	edge node {$*_2$} (fxy)
	
	(g) edge node {$\ma$} (ga)
	edge node {$*_1$} (gx)
		
	(ga) edge node {$*_1$} (gax)
	(gx) edge node {$\ma$} (gxa)
	
	;
\end{tikzpicture}\\[1em]

$\mh(\mf(y,x)) \mapsto \mh.\mf.{*}.{*}$\hspace{6em}$\mh(\mf(y,x)) \mapsto' \mh.\mf.{*_1}.{*_2}$
\end{center}
\end{frame}

%=== example forall ===


\begin{frame}
\begin{tikzpicture}
\node {$\exists $}
% [clockwise from=-170,sibling angle=-160]
child {node {$[a,b,c]$}}
child {node {${\mathsf p}$}
% [clockwise from=-90,sibling angle=0]
 child {node {$a$}}};
     \end{tikzpicture}
     
$(\exists [a,b,c]:({\mathsf p}(a)))$
\end{frame}

\begin{frame}
${\mathsf f}({\mathsf a},x)$

\begin{tikzpicture}
\node {${\mathsf f}$}
child {node {${\mathsf a}$}}
child {node {$x$}};
 \end{tikzpicture}

${\mathsf f}({\mathsf a},x)\approx {\mathsf f}(x,{\mathsf a})$

\begin{tikzpicture}
\node {$\approx $}
% [clockwise from=-150,sibling angle=-120]
child {node at (-0.7,1) {${\mathsf f}$}
 [clockwise from=-130,sibling angle=-80]
 child {node {${\mathsf a}$}}
 child {node {$x$}}}
child {node at (0.7,1){${\mathsf f}$}
 [clockwise from=-130,sibling angle=-80]
 child {node {$x$}}
 child {node {${\mathsf a}$}}};
 \end{tikzpicture}
\end{frame}
 
\end{document}
% ********************************
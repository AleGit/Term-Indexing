% !TEX encoding = UTF-8 Unicode

\documentclass[ 
% amsmath=
% amsthm=
% array=
% calc
% enumerate=
% extsizes
% geometry
% hyperref
xcolor={usenames,dvipsnames,svgnames,tablem} 
%,bigger
,handout
]{beamer}
% automatically loaded: xcolor, amsmath, amsthm, calc, geometry, hyperref, extsizes

% !TeX root = ../m3talk.tex
% !TeX encoding = UTF-8
% !TeX spellcheck = en_US

% !TeX root = ../m3talk.tex
% !TeX encoding = UTF-8
% !TeX spellcheck = en_US

%% ==============================================================
%% ### BEAMER ###

\usetheme[]{CambridgeUS}	
\usecolortheme{seagull}			% seahorse, fly, dolphin, dove, beetle, seagull ...
\useinnertheme{circles}			% tree, smoothtree, infolines, smoothbars	
\usefonttheme{professionalfonts} 	% professionalfonts serif structurebold structureitalicserif structuresmallcapsserif
	
\beamertemplatenavigationsymbolsempty

%% ==============================================================
%% ### "PACKAGES" ###

% BY BEAMER: xcolor, amsmath, amsthm, calc, geometry, hyperref, extsizes

\usepackage[utf8x]{inputenc} 								% Eingabekodierung	
\usepackage[polutonikogreek,ngerman,english]{babel}	
\usepackage[T1]{fontenc}	 								% Ausgabekodierung (PDF)
%\usepackage[usenames,dvipsnames,svgnames,table]{xcolor}		% BEAMER, Farben
%\usepackage{amsmath}									% BEAMTER
%\usepackage{amsthm}									% BEAMTER
\usepackage{amssymb}									% \ngtpreq

\usepackage{proof} 				%\infer, \deduce
%\usepackage{marvosym}			% \Lightning

%\usepackage{soul}			% \so\caps\ul\st\hl
%\usepackage[normalem]{ulem} % \uline\uuline\sout\xout\uwave
%\usepackage{transparent}
%\usepackage{listings}
%\usepackage{multirow}
%\usepackage{pdfpages}
%\usepackage[weather]{ifsym} % does not work with tex writer
%\let\Sun\relax % defined in marvosym too
%\let\Lightning\relax % defined in marvosym too

%\usepackage{ulsy}	% \blitza \blitzb ... \blitze (does not exist anymore? just not included?)
% !TEX root = ../m3talk.tex
% !TEX encoding = UTF-8 Unicode

%% color definitions

%\colorlet{col:a}{Fuchsia}
%\colorlet{col:b}{Blue}
\colorlet{colG}{Gray}
\colorlet{colO}{Orange}
\colorlet{colHi}{Green}
\colorlet{colLo}{Red}
\colorlet{colNa}{Gray}
\colorlet{colN}{Blue}
\colorlet{colEm}{RoyalBlue}
%\colorlet{col:max}{Violet}
%\colorlet{col:smx}{RoyalBlue}

%\newcommand{\colA}{\color{col:a}}	% example a
%\newcommand{\colB}{\color{col:b}}	% example b
\newcommand{\colG}{\color{colG}}	% neutral
%\newcommand{\colO}{\color{col:o}}	% old
\newcommand{\colN}{\color{colN}}	% new
\newcommand{\colHi}{\color{colHi}}	% hi, true
\newcommand{\colLo}{\color{colLo}}	% lo, false
\newcommand{\colNA}{\color{colNa}}	% not available		
%\newcommand{\colMAX}{\color{col:max}}	% maximal
%\newcommand{\colSMX}{\color{col:smx}}	% strictly maximal		

%\newcommand{\MYa}{\color{Fuchsia}}
%\newcommand{\MYb}{\color{Blue}}
%\newcommand{\MYg}{\color{Gray}}
%\newcommand{\MYo}{\color{Orange}}
%
%\newcommand{\MYhi}{\color{Green}}
%\newcommand{\MYlo}{\color{Red}}
%\newcommand{\MYna}{\color{Gray}}
%
%\newcommand{\MYA}[1]{{\MYa#1}}
%\newcommand{\MYB}[1]{{\MYb#1}}
%\newcommand{\MYG}[1]{{\MYg#1}}
%\newcommand{\MYO}[1]{{\MYo#1}}
%
%\newcommand{\MYHI}[1]{{\MYhi#1}}
%\newcommand{\MYLO}[1]{{\MYlo#1}}
%\newcommand{\MYNA}[1]{{\MYna#1}}
%
%\newcommand{\MYS}[1]{{\color{Violet}#1}}
%\newcommand{\MYM}[1]{{\color{RoyalBlue}#1}}
%
%\newcommand{\mLightning}{{\text{\Lightning}}}

\newcommand{\BEGINX}{\begin{example}}
	\newcommand{\ENDX}{\end{example}}

\newcommand{\BEGIN}{\begin{exampleblock}}
\newcommand{\END}{\end{exampleblock}}
% !TEX root = ../m3talk.tex
% !TEX encoding = UTF-8 Unicode

%% ==============================================================
%% ### MISC ###

%\newcommand{\EQ}{\simeq}
%\newcommand{\NEQ}{\not\simeq}
\newcommand{\foEQ}{\approx}		%	{\simeq}
\newcommand{\foNEQ}{\not\foEQ}

%\newcommand\mytop[2]{\genfrac{}{}{0pt}{}{#1}{#2}}
%\newcommand{\mygreek}[1]{\selectlanguage{polutonikogreek}#1\selectlanguage{english}}
%\newcommand{\mygreek}[1]{{\selectlanguage{polutonikogreek}#1}\selectlanguage{english}}
%\renewcommand{\mygreek}[1]{\foreignlanguage{polutonikogreek}{#1}}

%\newcommand{\iSUB}[2]{#2\!\mapsto\!#1}
%\newcommand{\BgSyntaxTree}{\usebackgroundtemplate{\transparent{0.1}\includegraphics[width=\paperwidth]{SyntaxTreeBackground.png}}}

\newcommand{\EMPH}[1]{\emph{\textcolor{colEm}{#1}}}

%% ==============================================================
%% ### MY MATH ENVIRONMENTS ###

\theoremstyle{plain}
%\newtheorem{theorem}{Theorem}
%\newtheorem{proposition}{Proposition}
%\newtheorem{lemma}{Lemma}
%\newtheorem*{corollary}{Corollary}

\theoremstyle{definition}
%\newtheorem{definition}{Definition}
\newtheorem{Conjecture}{Conjecture}
%\newtheorem*{example}{Example}
%\newtheorem{algorithm}{Algorithm}
\newtheorem{procedure}{Procedure}
\newtheorem{notation}{Notation}

\theoremstyle{remark}
\newtheorem*{remark}{Remark}
%\newtheorem*{note}{Note}
%\newtheorem{case}{Case}

%% ==============================================================
%% ### MY MATH DEFINITIONS ###

% math operators

\DeclareMathOperator{\arity}{arity}
\DeclareMathOperator{\var}{var}
\DeclareMathOperator{\pos}{pos}
\DeclareMathOperator{\T}{T}
\DeclareMathOperator{\dom}{dom}
\DeclareMathOperator{\rng}{rng}
\DeclareMathOperator{\img}{img}
\DeclareMathOperator{\mgu}{mgu}
\DeclareMathOperator{\wgt}{W\!}
\DeclareMathOperator{\sel}{sel}
\DeclareMathOperator{\mul}{mul}
\DeclareMathOperator{\add}{add}

\DeclareMathOperator{\unif}{unifiable}
\DeclareMathOperator{\inst}{instance}
\DeclareMathOperator{\gene}{generalization}
\DeclareMathOperator{\vari}{variant}

% constant (function, predicate) symbols

\newcommand{\mf}{{\mathsf f}}
\newcommand{\mg}{{\mathsf g}}
\newcommand{\mh}{{\mathsf h}}
\newcommand{\ma}{{\mathsf a}}
\newcommand{\mb}{{\mathsf b}}
\newcommand{\mc}{{\mathsf c}}
\newcommand{\md}{{\mathsf d}}
\newcommand{\mx}{{\mathsf x}}
\newcommand{\my}{{\mathsf y}}
\newcommand{\msucc}{{\mathsf s}}
\newcommand{\mpred}{{\mathsf p}}
\newcommand{\mA}{{\mathsf A}}
\newcommand{\mB}{{\mathsf B}}
\newcommand{\mP}{{\mathsf P}}
\newcommand{\mQ}{{\mathsf Q}}

% caligraphic symbols

\newcommand{\mcC}{{\mathcal C}}
\newcommand{\mcD}{{\mathcal D}}
\newcommand{\mcE}{{\mathcal E}}
\newcommand{\mcF}{{\mathcal F}}
\newcommand{\mcM}{{\mathcal M}}
\newcommand{\mcR}{{\mathcal R}}
\newcommand{\mcT}{{\mathcal T}}
\newcommand{\mcV}{{\mathcal V}}

\newcommand{\VAR}{\mcV\mathtt{ar}}
% fraktal symbols

\newcommand{\mfC}{{\mathfrak C}}
\newcommand{\mfL}{{\mathfrak L}}
\newcommand{\mfR}{{\mathfrak R}}
\newcommand{\mfT}{{\mathfrak T}}

% tt symbols

\newcommand{\mtS}{{\mathtt S}}
\newcommand{\sgr}{\succ_{\mathsf gr}}

% 

\newcommand{\joins}{\rightarrow^*\cdot^*\!\!\leftarrow}
\newcommand{\meets}{^*\!\!\leftarrow \cdot \rightarrow^* }


\newcommand{\mCP}[1]{\mathsf{CP}(#1)}		% Critical Pair
\newcommand{\mCPR}{\mCP{\mcR}}		% CP(R)

\newcommand{\MUL}[2]	% multiplication
{\mf(#1,#2)}			% mul(x,y)
%{#1\cdot #2}			% x.y

\newcommand{\ADD}[2]	% addition
{\add(#1,#2)}			% add(x,y)
%{#1+#2}				% x+y

\newcommand{\MYPOS}[1]{{\tt #1}}
\newcommand{\overlap}[3]{\langle #1,\MYPOS{#2}, #3 \rangle}
\newcommand{\overlapN}[4]{{_{\overlap{#1}{#2}{#3}}}^{#4:\;}}

%\newcommand{\GTKBO}{>_{\tt kbo}}
\newcommand{\GTKBOW}[2]{\texttt{SMT}(#1\!>_\texttt{kbo}\!#2)}
\newcommand{\GTKBOP}[2]{\texttt{SMT}(#1\!>_\texttt{kbo}'\!#2)}

\newcommand{\UPL}{\infer
	[(\sigma)
		\quad\sigma=\mgu(l,l'), l'\!\not\in\mcV, l\sigma\rho\sgr r\sigma\rho
	]
	{L[r]\sigma}
	{l=r & L[l']}	
}

\newcommand{\emptyclause}{\square}

% !TeX root = ../m3talk.tex
% !TeX encoding = UTF-8
% !TeX spellcheck = en_US

%% ### TIKZ ###

\usepackage{tikz}
\usetikzlibrary{automata,arrows,shapes,trees,decorations.pathmorphing
% snakes have been supersede by decorations
%arrows,shapes,,automata,backgrounds,petri, positioning
%,,automata,shadows,fit,
% decorations,pathmorphing
% graphs
}

\tikzset{
%	->,
%	>=stealth', 
%	shorten >=1pt, 
%	auto,
	node distance=2.1cm, 
%	semithick,
	minimum size=0,
	inner sep=1,
	outer sep=1mm,
%
	initial text=$\varepsilon$,
%	
	every state/.style={
%		fill=red,
%		draw=none,
%		text=white,
		radius=0.1em
	},
	my/.style={ rectangle, draw=red	},
%	sloped,below
}


%\tikzstyle{myrect} = [rectangle,draw=black,rounded corners, minimum height=3em, thick, text centered,text width=5.5em]
%\tikzstyle{mykaro} = [diamond,draw=black,rounded corners, thick, text centered,text width=4em]
%\tikzstyle{mycircle} = [circle,draw=black,thick, text centered, minimum height=3.5em,text width=4em,text width=3.5em]
%\tikzstyle{myarrow} = [thick,->,>=stealth]
%
%\tikzstyle{myframe} = [rectangle,draw=black,rounded corners, minimum height=3em, thick, text centered,text width=5.5em]

%% ==============================================================

%% ==============================================================






% ********************************

\author{Alexander Maringele}
\title{Term-Indexing
%\\and\\First Order Theorem Proving
}
\subtitle{for Instantiation-Based\\First Order Theorem Proving}
\date{January 27th, 2016}
%======

\begin{document}

\selectlanguage{english}

%\Cloud

% === TITLE ===
\frame[<+->]{
\maketitle
}

% === REFERENCES ===
\section*{References}
\frame[<+->]{
\frametitle{References}

\nocite{SRV2001ti,RV2003eir}	% ZHM2009jar
\bibliographystyle{amsalpha}
\bibliography{biblio}

}

% === OVERVIEW ===

\section*{Outline}
\setcounter{tocdepth}{1}
\frame[<+->]{
\frametitle{Outline}
	\tableofcontents
}



%====================================================================
% BEGIN: CONTENT ----------------------------------------------------
%====================================================================

%hallo

\section{Motivation}
\subsection{Notion}

\subsection{First Order Clause Logic}

\begin{frame}

\begin{notation}[First Order]
\begin{enumerate}
\item $\Sigma = (\SIGV, \SIGF, \SIGP)$ \hfill signature
%\item $P \in \Sigma_\mP = \{ \mA, \mB, \ldots \}$ \hfill predicate symbols 
%\item $F \in \Sigma_\mf = \{ \mc, \mf,\mg,\ldots \}$ \hfill (constant) function symbols
%\item $V = \{ x,y,z, \ldots\} $ \hfill variables
%\item $C ::=  \lnot \mid \lor \mid {\colG \land \mid\ \rightarrow}$\hfill connectives
%\item {\colG $Q ::= \forall, \exists$ \hfill quantifiers}
\item $\SIGT = \SIGV \cup \{ f(t_1,\ldots,t_n)\mid f\in\SIGF_{(n)}, t_i \in \SIGT \}$\hfill terms
\item $\SIGA = \{ P ( t_1, \ldots, t_n )\mid P\in\SIGP_{(n)}, t_i \in \SIGT\}
\cup
\{ s \foEQ t \mid s,t\in \SIGT\}$\hfill  atoms
%\item {\colG $F ::= A \mid (\lnot F) 
%	\mid (F \lor F)  \mid (F \land F)  
%	\mid (F \rightarrow F)
%	\mid (\forall F)
%	\mid (\exists F) 
%	$ 
%	\hfill formulas
%	
%	}
\item $\SIGL = \SIGA \cup \{ \lnot A \mid A\in \SIGA \}$\hfill literals
\item $\SIGC = 2^\SIGL$, e.g. $\mf(x)\foEQ\ma \lor \ma\foNEQ\mb$\hfill clauses
\item $\SIGS = 2^\SIGC$, e.g. $\{ \mf(x)\foEQ\ma \lor \ma\foNEQ\mb, \mf(x)\foEQ\mb\}$\hfill clause sets
\end{enumerate}
\[
\{ C_1, \ldots, C_n \} \equiv 
\forall\Var(C_1) C_1 
\land \ldots \land
\forall\Var(C_1) C_n 
\]
%A set of clauses is equivalent to a conjunction of universally quantified disjunctions of literals
%and equivalent to a universally quantified conjunction of variable distinct disjunctions of literals.

%\item $S ::= \{ C,\lots,C \}$\hfill sets of clauses
%\item Formulae $F::=$


\end{notation}
\end{frame}

% === saturation-based theorem provers ===
\subsection{Resolution-based theorem provers}
\frame{
	\frametitle{Resolution-based theorem provers}
	\begin{procedure}	
	\begin{enumerate}
	\item Transform the negation of a conjecture $F$
	
	into a equisatisfiable set of clauses $S$.
	\item Expand $S$ with derived clauses 
	in a sufficient$^{\circ\diamond\star}$ way.
	\item Stop if 
	\begin{enumerate}
	\item a proof for unsatisfiability of $S$ has been found\hfill{\colG F is a theorem}
	\item the set $S$ is saturated, hence satisfiable\hfill{\colG F is not a theorem} 
	\item time's up or space's out\hfill \colG we don't know
	\end{enumerate}
	Otherwise, continue with 2.
	\end{enumerate}
	\end{procedure}
	
	\begin{goal}[]
	Sound$^\circ$, refutational complete$^\diamond$, and effective$^\star$ procedure.
	\end{goal}
}

% === the big picture === 
\begin{frame}

\begin{tikzpicture}[scale = 1, transform shape, draw=black, fill=black, thick, sloped]

	\draw[->, ultra thick] (0,0) -- 
%	node[pos=0, above] {$\lnot F\approx S$} 
	node[pos=0, above] {$F$} 
	node[pos=0.32, below] {\color{colN}$\lnot F \approx\ S$}
	node[pos=1.25] {$S$} 
	(2,0);
		
	% outer rectangle
	\draw[rounded corners=-1.5mm] (1,3) rectangle (8.5,-3);
	% is F a theorem?
	\draw(2.4,2.7) node {\color{colN}Is $F$ a theorem?};
	
	% inner rectangle
	\draw[rounded corners=-1.5mm]  (1.5,-2.25) rectangle (8,2.25); 
	 % is S satisfaible?
	 \draw (2.9,-1.9) node {\color{colN}Is $S$ satisfiable?};
	  
	 \draw[thin,dashed,draw=colO] (2.5,0) ellipse (0.4 and 1.2); 
	 \draw[thin,dashed,draw=colO] (2.6,0) ellipse (0.6 and 1.44); 
	  \draw[thin,dashed,draw=colO] (2.8,0) ellipse (0.9 and 1.73); 
	  \draw[thin,dashed,draw=colO] (3.2,0) ellipse (1.35 and 2.07);
	  
	\draw[dashed, decoration=snake, decorate,draw=colG, thick] (1.4,-1) -- (8.1,-0.6);
	\draw[dashed, decoration=snake, decorate,draw=colG, thick] (1.4, 1) -- (8.1,0.6);
		
	\draw[->, draw=colHi, ultra thick] (6.5,1.8) -- 
	node[pos=0,below] {unsatisfiable}
	node[pos=0.85, above] {theorem} 
	(10,1.8) ;
	
	\draw[->,draw=colNa, ultra thick] (7,0.15) -- 
%	node[pos=0,above] {space out}
	node[pos=0,below] {time out}
	node[pos=0.85, above] {maybe} (10.5,0.15) ;
	
	\draw[->,draw=colLo, ultra thick] (7,-1.3) -- 
	node[pos=0, below] {satisfiable}
	node[pos=0.75, above] {not a theorem} (11,-1.3) ;
\end{tikzpicture}

\end{frame}

% === Example Superpostition ===
%\subsection{Superposition}
%\begin{frame}
%
%\begin{example}[Superposition]
%\vspace{-1em}
%\begin{gather*}
%S = \{ {}^{1:}\mf(\mh(x))\foEQ\mc \lor \mh(\mh(x))\foNEQ \ma, 
%	{}^{2:}\mh(y)\foEQ y, 
%	{}^{3:}\mf(\ma)\foNEQ \mc \}\qquad\\[0.5em]
%\newcommand{\stepA}{
%	\infer[ \alpha % (S_+)\ \{ y\mapsto x \}
%	]{	\mf(x)\foEQ\mc \lor {\colNA\mh(\mh(x))\foNEQ \ma}
%	}{	^{2:}\mh(y)\foEQ y
%		& 
%		^{1:}\mf(\boxed{\mh(x)})\foEQ\mc \lor {\colNA\mh(\mh(x))\foNEQ \ma}
%	}
%}
%\newcommand{\stepB}{
%	\infer[	\beta]
%	{		{\colNA\mh(\mh(\ma))\foNEQ \ma} \lor \mc\foNEQ\mc
%	}{		\stepA & ^{3:}\boxed{\mf(\ma)}\foNEQ \mc
%	}
%}
%\newcommand{\stepC}{
%	\infer[\{\}
%	]{		\mh(\boxed{\mh(\ma)})\foNEQ \ma
%	}{		\stepB
%	}
%}
%\newcommand{\stepD}{
%	\infer[\gamma
%	]{	\boxed{\mh(\ma)}\foNEQ\ma
%	}{	^{2:}\mh(y)\foEQ y\hspace{-3cm} & \stepC
%	}
%}
%\newcommand{\stepE}{
%	\infer[\gamma
%	]{	\ma\foNEQ\ma
%	}{	^{2:}\mh(y)\foEQ y & \stepD
%	}
%}
%\newcommand{\stepF}{
%	\infer[\{\}	%(f)
%	]{	\emptyclause
%	}{	\stepE
%	}
%}
%\stepF
%\end{gather*}
%\center
%$\alpha = \{ y\mapsto x \}$, 
%$\beta=\{ x\mapsto\ma\}$, 
%$\gamma=\{ y\mapsto\ma\}$ 
%\end{example}
%\end{frame}

% === Example Inst-Gen-Eq ===
\subsection{Inst-Gen-Eq}
\begin{frame}
\begin{example}[Inst-Gen-Eq]
%
\vspace{-1em}
%
\newcommand{\inferFXC}{
		\infer[ \sigma]
		{ \mf(x)\foEQ\mc }
		{ ^{2^\ell:}\mh(y)\foEQ y & ^{1^\ell:}\mf(\boxed{\mh(x)})\foEQ\mc} 	
	}
	\newcommand{\inferCnC}{
		\infer[ \tau ]
		{ \emptyclause }
		{ \inferFXC & ^{3^\ell:}\mf(\ma)\foNEQ \mc } 		
	}
%
	\begin{align*}
	S_0 &= \{ ^{1:}\mf(\mh(x))\foEQ\mc \lor \mh(\mh(x))\foNEQ \ma, 
	^{2:}\mh(y)\foEQ y, 
	^{3:}\mf(\ma)\foNEQ \mc \} 
	\\
	S_0\bot &= \{ 
	{\colHI\mf(\mh(\bot))\foEQ\mc} \lor 
	{\colNA\mh(\mh(\bot))\foNEQ \ma}, 
	{\colHI\mh(\bot)\foEQ\bot}, 
	{\colHI\mf(\ma)\foNEQ\mc} 
	\} 
\\
	&
	{\inferCnC
	} 
\\
	S_1 &= \{ \ldots, 
		\mf(\ma)\foNEQ\mc, 
		^{2\sigma\tau:}{\colN\mf(\mh(\ma))\foEQ\mc \lor \mh(\mh(\ma))\foNEQ \ma}, 
		^{3\tau:}{\colN\mh(\ma)\foEQ\ma}
	\} 
\\
	S_1\bot &=
 	\{ \ldots,
		{\colHI\mf(\ma)\foNEQ \mc}, 
 		{\colLO \mf(\mh(\ma))\foEQ\mc
		\lor\mh(\mh(\ma))\foNEQ \ma}, 
		{\colHI\mh(\ma)\foEQ\ma} 
	\} 
	\end{align*}
%
\center
$\sigma=\{ y\mapsto x \}$, $\tau=\{ x\mapsto \ma \}$
\end{example}

%\begin{remark}
%	Superposition [Bachmair, Ganzinger],
%	InstGenEq (Ganzinger, Korovin) are sound and refutational complete with a \EMPH{fair} strategy.
%	\end{remark}
\end{frame}

% === infinite domain ===

%\begin{frame}
%\begin{align*}
%S_1 &= \{ 
%	\mf(x,x)\foNEQ a,
%	\mf(y,\mg(y)) \foEQ a,\\
%	&\qquad\mf(x,y)\foNEQ a \lor \mf (y,z)\foNEQ a \lor \mf(x,z)\foEQ a 
%	\}
%	\\
%%S_1\bot &= \mf(\bot,\bot)\foNEQ a,
%%	\mf(\bot,\mg(\bot)) \foEQ a,
%%	\mf(\bot,\bot)\foNEQ a \lor \mf (\bot,\bot)\foNEQ a \lor f(\bot,\bot)\foEQ a \\
%\end{align*}
%\end{frame}

% === Term retrieval problems ===

\frame{
\frametitle{Term retrieval problems}
\begin{itemize}
\item Find terms that are variants of a given term.
$\vari(\ell,t) \Leftrightarrow \exists\sigma\ \ell\sigma = t$ and $\sigma$ is renaming.
\item Find terms that are unifiable with a given term. 
$\unif(\ell,t) \Leftrightarrow \exists\sigma\ \ell\sigma = t\sigma$
\item Find terms that are instances of a given term.
$\inst(\ell,t) \Leftrightarrow \exists\sigma\ \ell = t\sigma$
\item Find terms that are generalizations of a given term.
$\gene(\ell,t) \Leftrightarrow \exists\sigma\ \ell\sigma = t$
\end{itemize}
\begin{definition}
\end{definition}
}

% === Definition Position ===

\subsection{Position}
\begin{frame}
% p.1860, 2.1 Definition
% 703811, p.26, 2.1.14
\begin{definition}
A position is a sequence of positive integers.
The empty sequence $\varepsilon$ denotes the root position,
$pq$ denotes the concatenation of positions. 
%$p$ and $q$,
%The set of positions is recursively defined: 
%\[
%\pos(t)=\left\{ 
%\begin{array}{ll}
%	\{ \varepsilon \} & \text{if }t\text{ is variable}\\
%	\{ \varepsilon \} \cup \{ ip\mid 1\leq i\leq n \land p \in \pos(t_i) \} & \text{if }t=\mf(t_1,\ldots,t_n)
%\end{array}
%\right.
%\]
%The subterm of $t$ at position $p\in\pos(t)$ is defined:
%\[
%t|_p = \left\{
%\begin{array}{ll}
%	t & \text{if }p=\varepsilon\\
%	t_i|_q & \text{if }t=\mf(t_1,\ldots,t_n) \text{ and } p=iq
%\end{array}
%\right.
%\]
$\Pos(t)$ denotes the set of positions in term $t$, and $t|_p$ denotes the subterm of $t$ at positon $p\in\Pos(t)$.
\end{definition}

\begin{definition}
A postion string is a nonempty string of the form $\langle p_1,s_1\rangle\ldots\langle p_n,s_n\rangle$
where $p_i$ are positions and $s_i$ are function or variable symbols and
\begin{enumerate}
\item
if $p_i$ is a proper prefix of $p_j$ then $i<j$
\item
\end{enumerate}
\end{definition}
\end{frame}

% === ====

\begin{frame}
% 4.1 Position strings, 4.1 Defintion
\begin{definition}

\end{definition}
\end{frame}

% === state machine ====

%\begin{frame}
%\begin{tikzpicture}[->,>=stealth',shorten >=1pt,auto,node distance=2.1cm,
%                    semithick]
%  \tikzstyle{every state}=[fill=red,draw=none,text=white]
%
%  \node[initial,state] (A)                    {$q_a$};
%  \node[state]         (B) [above right of=A] {$q_b$};
%  \node[state]         (D) [below right of=A] {$q_d$};
%  \node[state]         (C) [below right of=B] {$q_c$};
%  \node[state]         (E) [below of=D]       {$q_e$};
%
%  \path (A) edge              node {0,1,L} (B)
%            edge              node {1,1,R} (C)
%        (B) edge [loop above] node {1,1,L} (B)
%            edge              node {0,1,L} (C)
%        (C) edge              node {0,1,L} (D)
%            edge [bend left]  node {1,0,R} (E)
%        (D) edge [loop below] node {1,1,R} (D)
%            edge              node {0,1,R} (A)
%        (E) edge [bend left]  node {1,0,R} (A);
%\end{tikzpicture}
%
%\href{http://www.texample.net/tikz/examples/state-machine/}{www.texample.net/tikz/examples/state-machine/}
%\end{frame}

\begin{frame}

\begin{gather*}
%
		\TI{1}\mh(\mf(x,x)),
		\TI{2}\mh(\mg(\ma,x)),
		\TI{3}\mh(\mf(y,z)),
		\TI{4}\mh(\mg(\ma,y)),
		\TI{5}\mh(\mf(y,x)),
		\TI{6}\mh(\mg(y,a))
		\\[0.5em]
%
\ps(t) =
\begin{cases}
	* &\text{if $t=x\in\SIGV$} \\
	\mf.\ps(t_1).\ps(t_2). \cdots.\ps(t_n) &\text{if $t=f(t_1,\ldots,t_n)$}
	\end{cases}
%
\end{gather*}
%	
\begin{center}
% imperfect filtering	
\begin{tikzpicture}[->,sloped,above]

\node[] (root) at (0,0.5) {.};

\node[] (h) at (0,-0.5) {.};
\node[] (f) at (-1,-1)  {.};
\node[] (g) at (1,-1) {.};

\node[] (fx) at (-1,-2) {.};
\node[] (fxx) at (-1,-3.3) {$\{ 1,3,5 \}$ };

\node[] (ga) at (1.6,-2)  {.};
\node[] (gax) at (1.6,-3.3) {$\{ 2,4 \}$};

\node[] (gx) at (.4,-2) {.};
\node[] (gxa) at (.4,-3.3) {$\{ 6 \}$};

\path (root) edge node {$\mh$} (h)

	(h) edge node {$\mf$} (f) 
	edge node {$\mg$}(g)
	
	(f) edge node {$*$} (fx)
	(fx) edge node {$*$} (fxx)
	
	(g) edge node {$\ma$} (ga)
	edge node {$*$} (gx)
		
	(ga) edge node {$*$} (gax)
	(gx) edge node {$\ma$} (gxa)
	
	;
\end{tikzpicture}
%
\hspace{1em}
%
% perfect filtering
\begin{tikzpicture}[->,sloped,above,node distance=1cm]

\node[] (root) at (0,0.5) {.};

\node[] (h) at (0,-0.5) {.};
\node[] (f) at (-1,-1)  {.};
\node[] (g) at (1,-1) {.};

\node[] (fx) at (-1,-2) {.};
\node[] (fxy) at (-1.6,-3.3) {$\{ 3,5 \}$ };
\node[] (fxx) at (-0.4,-3.3) {$\{ 1 \}$ };

\node[] (ga) at (1.6,-2)  {.};
\node[] (gax) at (1.6,-3.3) {$\{ 2,4 \}$};

\node[] (gx) at (.4,-2) {.};
\node[] (gxa) at (.4,-3.3) {$\{ 6 \}$};

\path (root) edge node {$\mh$} (h)

	(h) edge node {$\mf$} (f) 
	edge node {$\mg$}(g)
	
	(f) edge node {$*_1$} (fx)
	(fx) edge node {$*_1$} (fxx)
	edge node {$*_2$} (fxy)
	
	(g) edge node {$\ma$} (ga)
	edge node {$*_1$} (gx)
		
	(ga) edge node {$*_1$} (gax)
	(gx) edge node {$\ma$} (gxa)
	
	;
\end{tikzpicture}\\[0.7em]
%
$\ps(\mh(\mf(y,x))) = \mh.\mf.{*}.{*}$\hspace{2em}$\ps'(\mh(\mf(y,x))) = \mh.\mf.{*_1}.{*_2}$
\end{center}
\end{frame}

%=== example forall ===


\begin{frame}
\begin{tikzpicture}
\node {$\exists $}
% [clockwise from=-170,sibling angle=-160]
child {node {$[a,b,c]$}}
child {node {${\mathsf p}$}
% [clockwise from=-90,sibling angle=0]
 child {node {$a$}}};
     \end{tikzpicture}
     
$(\exists [a,b,c]:({\mathsf p}(a)))$
\end{frame}

\begin{frame}
${\mathsf f}({\mathsf a},x)$

\begin{tikzpicture}
\node {${\mathsf f}$}
child {node {${\mathsf a}$}}
child {node {$x$}};
 \end{tikzpicture}

${\mathsf f}({\mathsf a},x)\approx {\mathsf f}(x,{\mathsf a})$

\begin{tikzpicture}
\node {$\approx $}
% [clockwise from=-150,sibling angle=-120]
child {node at (-0.7,1) {${\mathsf f}$}
 [clockwise from=-130,sibling angle=-80]
 child {node {${\mathsf a}$}}
 child {node {$x$}}}
child {node at (0.7,1){${\mathsf f}$}
 [clockwise from=-130,sibling angle=-80]
 child {node {$x$}}
 child {node {${\mathsf a}$}}};
 \end{tikzpicture}
\end{frame}
 
\end{document}
% ********************************